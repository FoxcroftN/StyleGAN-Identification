\chapter{Reflection}

While completing the proposed project I learned a lot about Deep Learning and CNN. The problem of identifying StyleGAN images with the use of Deep Learning and CNN directed me towards learning about how the problem can be solved using technologies such as TensorFlow, Keras and Optuna to train a final model that can identify these generated images with high accuracy. The new technology Optuna increased my understanding of how neural networks function by forcing me to dive in deep and apply my foundational previous knowledge in the field of AI.

The strong point of the artefact created in this proposed project is the high accuracy in which StyleGAN images can be identified. The generalization of the final model is also sufficient that allows it to identify images that differ from the original dataset that was used. The font-end allows users to easily interact with such a complex method of detection without the need for any previous knowledge.

Weak points in the created Artefact is that the model can only detect StyleGAN1 images with high accuracy and falls short in detecting StyleGAN2 and StyleGAN3 generated images. The model sometimes misidentifying the images it receives can be seen as a weakness as when it misclassifies a user might use the misclassification for their decision-making process and in that sense, the artefact will aid in the passing of StyleGAN generated images as real images. The frequency of the above-mentioned weakness is so low that this possibility is negligible.

The aim of the proposed project was fully satisfied and the final method of detection can identify StyleGAN images consistently with high accuracy. The objectives of the project were satisfied in the study in StyleGAN and a full understanding of the technology that supported the development of the method of detection and the final implementation of the method. These objectives were all satisfied based on the high accuracy model being implemented in the frontend web application.

A  hurdle in the completion of the proposed project was the managing of the timeframe. The planning in the initial stages of the project allowed for all target dates to be met. The takeaway for me will be to plan for the unexpected as the increase in load-shedding times and other events cause some deviations of the original timeframe of the project.

The process followed in completing the project could be improved upon. The DSRM methodology was useful in the development of the artefact but a possible better methodology could be used to specifically cater for the machine learning project. Resource planning would have allowed for better training using larger datasets and more training steps that ultimately would have led to a model that would have been one of the best performing models for StyleGAN identification in current times. 